%@@@@@@@@@@@@@@@@@@@@@@@@@@@@@@
%@@@@@@@@@ Usepackage @@@@@@@@@
%@@@@@@@@@@@@@@@@@@@@@@@@@@@@@@
\documentclass{article}
% round numbers with space
\usepackage{numprint}
% key word
% \nplpadding[characters]{digits}
% \numprint{1234}


% setting margin
\usepackage[a4paper,left=0.5in,right=0.5in]{geometry}
% color package
\usepackage{xcolor}
% tablur combine
\usepackage{multirow}

%--------------------
%-- define comment --
%--------------------
%\long\def\/*#1*/{}
%__ define comment __
%____________________

%------------------
%-- math package --
%------------------
\usepackage{amssymb,amsmath,stmaryrd} 
%------------------
\usepackage{mathtools}
%Reference: 
%http://texdoc.net/texmf-dist/doc/latex/mathtools/mathtools.pdf
%__ math package __
%__________________


%------------------------------------
%-- insert {} to multiple folders ---
%------------------------------------
\usepackage{graphicx}
\graphicspath{{./images/}, {./image/}} 
%__ insert {} to multiple folders ___
%____________________________________


%------------------
%-- use chinese ---
%------------------
\usepackage{CJKutf8}
%__ use chinese ___
%__________________

%----------------------
%-- use draw package --
%----------------------
\usepackage{tikz}
%__ use draw package __
%______________________


%----------
%-- url ---
%----------
\usepackage{hyperref}
% sample
% \url{http://...}
% \href{http://...}{text hyperlink}
%__ url ___
%__________

%----------
%-- code --
%----------
\usepackage{listings}
%__ code __
%__________

%----------------
%-- underline ---
%----------------
\usepackage{soul}
% using with
% ul{...}
%__ underline ___
%________________


%----------------------
%-- background color --
%----------------------
\usepackage{lipsum}
\usepackage[most]{tcolorbox}
\definecolor{bg}{RGB}{255,249,227}
\definecolor{block_green}{RGB}{203,247,235}
%__ background color __
%______________________


%@@@@@@@@@@@@@@@@@@@@@@@@@@@@@@@@@@@@@@@@@@@@@@@@
%@@@@@@@@@@@@@@ Enumerate command @@@@@@@@@@@@@@@
%@@@@@@@@@@@@@@@@@@@@@@@@@@@@@@@@@@@@@@@@@@@@@@@@
\makeatletter
\newcommand\setItemnumber[1]{\setcounter{enum\romannumeral\@enumdepth}{\numexpr#1-1\relax}}
\makeatother

% command
%\setItemnumber{5}
%============== Enumerate command ===============
%================================================

%--------------------
% -- Merge pdf ------
\usepackage{pdfpages}

%\includepdf[pages=-]{paper1}

%--------------------
%--------------------

%----------------------------------------   multiple columns
\usepackage{multicol}

% begin{multicols}{2}
%
% end{multicols}
%---------------------

%------------------------------------------------------------
%
%               multiple divide add substract 
%
%------------------------------------------------------------
% add multiple
\usepackage{xlop}

% \opmul{384}{56}
% \opadd{1234}{567}

% divide
\usepackage{polynom}

% \input{longdiv}
% \longdiv{12345}{13}
%
%______________ multiple divide add substract _______________
%____________________________________________________________


%========= Usepackage =========
%==============================

%@@@@@@@@@@@@@@@@@@@@@@@@@@@@
%@@@@@@@@@ Document @@@@@@@@@
%@@@@@@@@@@@@@@@@@@@@@@@@@@@@
%------------
%-- Start ---
%------------
\title{Multiplicative Inverse\\(模數乘法反元素)\\}
\author{\textbf{sjLin}}
%\date{}
\begin{document}
{\fontfamily{lmss}\selectfont
% bsmi
\begin{CJK*}{UTF8}{bkai}
\maketitle
\noindent
%__ Start ___
%____________
假設$a\in \mathbb{Z},\; n\in \mathbb{Z}^+$,
$ax\equiv 1\pmod {n} $,則稱$x$為$a$在$\mod n$下的multiplicative inverse,而這個$x$有無限個,符合最小正整數$x$,稱為$a$在$\mod n$下的最小乘法反元素(least multiplicative inverse),這最小乘法反元素記作$a^{-1}\pmod {n} $。
\begin{tcolorbox}[enhanced jigsaw,colback=bg,boxrule=0pt,arc=0pt]
\section*{定理}
假設$a\in\mathbb{Z},\;n\in\mathbb{Z}^+$,若$\gcd(a,n)=1$, 則$a$在$\mod n$的乘法反元素存在。
\section*{證明}
因為$\gcd(a,n)=1$,所以$a,n$互質,使得$\exists s,t\in\mathbb{Z}$,則$as+nt=1$$\Rightarrow as+nt\equiv 1\pmod {n} $。因為$nt\equiv0\pmod {n} $,所以$as\equiv1\pmod {n} $。\\
因此$s$為$a$在$\pmod {n} $下的乘法反元素。
\end{tcolorbox}
\section*{例題-92 台大資工}
Find the inverse of $4$ modulo $7$.
\subsection*{Ans.}
$\exists s,t\in Z$,使得$4s+7t=1$。\\
由Eulidean Algorithm得知,\\	
$
7=1\cdot 4+3\\
4=1\cdot 3+1
$\\
$
\Rightarrow 
1=4-1\cdot 3\\
1=4-1(7-1\cdot 4)
$\\
整理\\
$
1=(-1)\cdot 7+2\cdot 4
$
\\
$\Rightarrow 1=(-1-4k)\cdot 7+(2+7k)\cdot 4,\;\forall k\in\mathbb{Z}$\\
因此,$2+7k,\forall k\in\mathbb{Z}$為$4$在$\pmod {7} $下的乘法反元素。
\section*{例題-98政大資料}
Find the least positive integer $x$ satisfying he congruence:
\[
	531x\equiv1\pmod {1769} 
\]
\subsection*{Ans.}
由Eulidean Algorithm得知,\\
$
1769=3\cdot 531+176\\
531=3\cdot 176+3\\
176=58\cdot 3+2\\
3=1\cdot 2+1
$\\
$
\Rightarrow 
1=3-2\\
=3-(176-58\cdot 3)\\
=3-176+58\cdot 3\\
=(-1)\cdot 176+59\cdot 3\\
=(-1)\cdot 176+59\cdot 531-177\cdot 176\\
=(-178)\cdot 176+59\cdot 53\\
=(-178)\cdot (1769-3\cdot 531)+59\cdot 531\\
=(-178)\cdot 1769+593\cdot 531
$\\
$\therefore 531^{-1}$在$\pmod {1769} $下的最小乘法反元素為593。\\
\rule{\textwidth}{0.5pt}
\begin{tcolorbox}[enhanced jigsaw,colback=bg,boxrule=0pt,arc=0pt]
\section*{推廣}
乘法反元素可以推廣到解一般的方程式$ax\equiv b\pmod {n} $
\end{tcolorbox}
\section*{例題-98清大資工}
Solve the linear congruence $7x\equiv 13\pmod {19} $
to find all the integer solutions $x$.
\subsection*{Ans.}
因為$\gcd(7,19)=1$,所以有整數解$(s,t)$,使得$7s+19t=1$,由Eulidean Algorithm得知,\\
$
19=2\cdot 7+5\\
7=1\cdot 5+2\\
5=2\cdot 2+1
$\\
$\Rightarrow 
1=5-2\cdot 2\\
=5-2(7-5)\\
=3\cdot 5+(-2)\cdot 7\\
=3(19-2\cdot 7)+(-2)\cdot 7\\
=3\cdot 19+(-8)\cdot 7
$\\
左右同乘$13$,即計算$7x\equiv13\pmod {19} $\\
$\Rightarrow 13=39\cdot 19+(-104)\cdot 7$\\
通式$13=(39-7k)\cdot 19+(-104+19k)\cdot 7,\;\forall k\in\mathbb{Z}$\\
$\Rightarrow -104\equiv 10\pmod {19} $\\
$\therefore 10+19k,\;\forall k\in\mathbb{Z}$為$x$的所有可能解。
\begin{tcolorbox}[enhanced jigsaw,colback=bg,boxrule=0pt,arc=0pt]
\section*{引理}
假設$a\in\mathbb{Z}$且$p$為一質數,則$a$為$a$在$\pmod {p} $下的乘法反元素$\leftrightarrow a\equiv \pm1\pmod {p} $
\subsection*{證明}
($\rightarrow $)\\
因為$a$為$a$在$\pmod {p} $下的乘法反元素,\\
$\Rightarrow a^2\equiv1\pmod {p} $\\
$p|(a^2-1)$\\
$p|(a-1)(a+1)$\\
所以$p|(a-1)$或$p|(a+1)$\\
因為同餘的兩數相減的值,會被$\mod$ 的值整除\\
所以$a\equiv -1\pmod {p} $或$a\equiv 1\pmod {p} $\\
($\leftarrow$)\\
因為$a\equiv\pm1\pmod {p} $,所以$a^2\equiv1\pmod {p} $\\
因此,$a$為$a$在$\pmod {p} $下的乘法反元素。
\end{tcolorbox}




%------------
%-- Finish --
%------------
\end{CJK*}
}
\end{document}
%__ Finish __
%____________

%========= Document =========
%============================

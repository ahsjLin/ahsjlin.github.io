%@@@@@@@@@@@@@@@@@@@@@@@@@@@@@@
%@@@@@@@@@ Usepackage @@@@@@@@@
%@@@@@@@@@@@@@@@@@@@@@@@@@@@@@@
\documentclass{beamer}

% setting margin
%\usepackage[a4paper,left=0.5in,right=0.5in]{geometry}
% color package
\usepackage{xcolor}
% tablur combine
\usepackage{multirow}

%--------------------
%-- define comment --
%--------------------
%\long\def\/*#1*/{}
%__ define comment __
%____________________

%------------------
%-- math package --
%------------------
\usepackage{amssymb,amsmath,stmaryrd} 
%------------------
\usepackage{mathtools}
%Reference: 
%http://texdoc.net/texmf-dist/doc/latex/mathtools/mathtools.pdf
%__ math package __
%__________________


%------------------------------------
%-- insert {} to multiple folders ---
%------------------------------------
\usepackage{graphicx}
\graphicspath{{./images/},{./image/}} 
%__ insert {} to multiple folders ___
%____________________________________


%------------------
%-- use chinese ---
%------------------
\usepackage{CJKutf8}
%__ use chinese ___
%__________________

%----------------------
%-- use draw package --
%----------------------
\usepackage{tikz}
%__ use draw package __
%______________________


%----------
%-- url ---
%----------
\usepackage{hyperref}
% sample
% \url{http://...}
% \href{http://...}{text hyperlink}
%__ url ___
%__________

%----------------
%-- underline ---
%----------------
\usepackage{soul}
% using with
% ul{...}
%__ underline ___
%________________

%----------------------
%-- background color --
%----------------------
\usepackage{lipsum}
\usepackage[most]{tcolorbox}
\definecolor{bg}{RGB}{255,249,227}
\definecolor{block_green}{RGB}{203,247,235}
%__ background color __
%______________________

%----------
%-- code --
%----------
\usepackage{listings}
%__ code __
%__________

%@@@@@@@@@@@@@@@@@@@@@@@@@@@@@@@@@@@@@@@@@@@@@@@@
%@@@@@@@@@@@@@@ Enumerate command @@@@@@@@@@@@@@@
%@@@@@@@@@@@@@@@@@@@@@@@@@@@@@@@@@@@@@@@@@@@@@@@@
\makeatletter
\newcommand\setItemnumber[1]{\setcounter{enum\romannumeral\@enumdepth}{\numexpr#1-1\relax}}
\makeatother

% command
%\setItemnumber{5}
%============== Enumerate command ===============
%================================================

%--------------------
% -- Merge pdf ------
\usepackage{pdfpages}

%\includepdf[pages=-]{paper1}

%--------------------
%--------------------

%----------------------------------------   multiple columns
\usepackage{multicol}

% begin{multicols}{2}
%
% end{multicols}
%---------------------
%------------------------------------------------------------
%
%               multiple divide add substract 
%
%------------------------------------------------------------
% add multiple
\usepackage{xlop}

% \opmul{384}{56}
% \opadd{1234}{567}

% divide
\usepackage{polynom}

% \input{longdiv}
% \longdiv{12345}{13}
%
%______________ multiple divide add substract _______________
%____________________________________________________________

%========= Usepackage =========
%==============================

%@@@@@@@@@@@@@@@@@@@@@@@@@@@@
%@@@@@@@@@ Document @@@@@@@@@
%@@@@@@@@@@@@@@@@@@@@@@@@@@@@
\title{\textbf{}}
\author{\textbf{sjLin}}
%\date{}
%------------
%-- Start ---
%------------
\begin{document}

{\fontfamily{lmss}\selectfont
% bsmi
\begin{CJK*}{UTF8}{bkai}

%\maketitle

%__ Start ___
%____________




\begin{frame}
\frametitle{Euclidean Algorithm}
歐幾里德演算法又稱輾轉相除法,即
\[
\begin{cases}
	\gcd(n,0)=0, \;\text{if } n>0 \\
	\gcd(m,n)=\gcd(n,m\mod n), \text{if } m>n
\end{cases}
\]
假設$m\geq n$,令$r_0=m,r_1=n$,不斷地利用毆幾里德演算法
\begin{align*}
	&r_0=q_1r_1+r_2,\; &0\leq r_2<r_1\\
	&r_1=q_1r_2+r_3,\; &0\leq r_3<r_2\\
	&\vdots\\	
	&r_{n-2}=q_{n-1}r_{n-1}+r_n,\; &0\leq r_n<r_{n-1}\\
	      &r_{n-1}=q_nr_n&
\end{align*}
\end{frame}

\begin{frame}
\frametitle{}
則$\gcd(m,n)=\gcd(r_0,r_1)=\gcd(r_1,r_2)=\dots=\gcd(r_{n-2},r_{n-1})=\gcd(r_{n-1},r_n)=\gcd(r_n,0)=r_n$
\end{frame}
\begin{frame}
\frametitle{例題}
\framesubtitle{94 中山資工}
題目:\\
Find all of the possible solutions of $250x+111y=7$, where both $x$ and $y$ are integers.\\[10pt]
Ans.
\begin{align*}
250&=2\times 111 +28\\
111&=3\times 28 + 27\\
28 &= 1\times 27 + 1
\end{align*}
從最後一個式子代回去。
\begin{align*}
1 &= 28 - 27\\
1 &= 28 - (111-3\times 28)\\
\text{整理}\\
1 &= (-1)\times111+4\times 28
\end{align*}
	
\end{frame}
\begin{frame}
\begin{align*}
	1 &= (-1)\times111+4\times (250-2\times 111)\\
	  &\text{整理}\\
	1 &= (-9)\times111+4\times 250\\
	  &\text{寫成通式(小的為負,大的為正)}\\	
	1 &= (-9+250k)\times111+(4-111k)\times 250,\;\forall k\in\mathbb{Z}\\
	  &\text{左右同乘}7\\
	7 &= 7(-9+250k)\times111+7(4-111k)\times 250,\;\forall k\in\mathbb{Z}
\end{align*}
所以$x=7(4-111k),\; y=7(-9+250k)$, $\forall k\in\mathbb{Z}$
	
\end{frame}





%------------
%-- Finish --
%------------
\end{CJK*}
}
\end{document}
%__ Finish __
%____________

%========= Document =========
%============================

%@@@@@@@@@@@@@@@@@@@@@@@@@@@@@@
%@@@@@@@@@ Usepackage @@@@@@@@@
%@@@@@@@@@@@@@@@@@@@@@@@@@@@@@@
\documentclass{beamer}

%\usetheme{Madrid}
%\usecolortheme{beaver}

% setting margin
%\usepackage[a4paper,left=0.5in,right=0.5in]{geometry}
% color package
\usepackage{xcolor}
% tablur combine
\usepackage{multirow}

%--------------------
%-- define comment --
%--------------------
%\long\def\/*#1*/{}
%__ define comment __
%____________________

%------------------
%-- math package --
%------------------
\usepackage{amssymb,amsmath,stmaryrd} 
%------------------
\usepackage{mathtools}
%Reference: 
%http://texdoc.net/texmf-dist/doc/latex/mathtools/mathtools.pdf
%__ math package __
%__________________


%------------------------------------
%-- insert {} to multiple folders ---
%------------------------------------
\usepackage{graphicx}
\graphicspath{{./images/},{./image/}} 
%__ insert {} to multiple folders ___
%____________________________________


%------------------
%-- use chinese ---
%------------------
\usepackage{CJKutf8}
%__ use chinese ___
%__________________

%----------------------
%-- use draw package --
%----------------------
\usepackage{tikz}
%__ use draw package __
%______________________


%----------
%-- url ---
%----------
\usepackage{hyperref}
% sample
% \url{http://...}
% \href{http://...}{text hyperlink}
%__ url ___
%__________

%----------------
%-- underline ---
%----------------
\usepackage{soul}
% using with
% ul{...}
%__ underline ___
%________________

%----------------------
%-- background color --
%----------------------
\usepackage{lipsum}
\usepackage[most]{tcolorbox}
\definecolor{bg}{RGB}{255,249,227}
\definecolor{block_green}{RGB}{203,247,235}
%__ background color __
%______________________

%----------
%-- code --
%----------
\usepackage{listings}
%__ code __
%__________

%@@@@@@@@@@@@@@@@@@@@@@@@@@@@@@@@@@@@@@@@@@@@@@@@
%@@@@@@@@@@@@@@ Enumerate command @@@@@@@@@@@@@@@
%@@@@@@@@@@@@@@@@@@@@@@@@@@@@@@@@@@@@@@@@@@@@@@@@
\makeatletter
\newcommand\setItemnumber[1]{\setcounter{enum\romannumeral\@enumdepth}{\numexpr#1-1\relax}}
\makeatother

% command
%\setItemnumber{5}
%============== Enumerate command ===============
%================================================

%--------------------
% -- Merge pdf ------
\usepackage{pdfpages}

%\includepdf[pages=-]{paper1}

%--------------------
%--------------------

%----------------------------------------   multiple columns
\usepackage{multicol}

% begin{multicols}{2}
%
% end{multicols}
%---------------------
%------------------------------------------------------------
%
%               multiple divide add substract 
%
%------------------------------------------------------------
% add multiple
\usepackage{xlop}

% \opmul{384}{56}
% \opadd{1234}{567}

% divide
\usepackage{polynom}

% \input{longdiv}
% \longdiv{12345}{13}
%
%______________ multiple divide add substract _______________
%____________________________________________________________

%--------------------   font
%\usepackage{lmodern}
\usepackage[T1]{fontenc}


%========= Usepackage =========
%==============================

%@@@@@@@@@@@@@@@@@@@@@@@@@@@@
%@@@@@@@@@ Document @@@@@@@@@
%@@@@@@@@@@@@@@@@@@@@@@@@@@@@
\title{\textbf{TOEIC}}
\author{\textbf{sjLin}}
%\date{}
%------------
%-- Start ---
%------------
\begin{document}

{\fontfamily{lmss}\selectfont
% bsmi
\begin{CJK*}{UTF8}{bkai}

\maketitle

%__ Start ___
%____________




\begin{frame}
\frametitle{聽力測驗\;-\;100題}

\begin{itemize}
\item 照片敘述\;-\;播放4個選項\;不會顯示\;6題
\item 應答問題\;-\;播放3個選項\;不會顯示\;25題
\item 簡短對話\;-\;每段對話有三小題\;每題3個選項\;會顯示\;13段對話\;39題
\item 簡短獨白\;-\;每段對話有三小題\;其中有兩段需要搭配圖表  共30題
\end{itemize}

\end{frame}

\begin{frame}
\frametitle{閱讀測驗\;-\;100題}
\begin{itemize}
\item 單句填空\;-\; 4個選項完成句子\;30題
\item 短文填空\;-\; 4篇短文\;每篇有4個空格\;4個選項填空\;16題
\item 閱讀測驗\;54題
\begin{itemize}
\item 單篇文章\;-\;2~4題\;共29題
\item 雙篇文章\;-\;每兩篇搭配5題\;共10題
\item 三篇文章\;-\;每三篇搭配5題\;共15題
\end{itemize}
\end{itemize}

\end{frame}

\section{閱讀測驗}
\begin{frame}
\frametitle{閱讀測驗\;注意}
單篇文章的最後或倒數第二題難度較難
\end{frame}
\begin{frame}
\frametitle{閱讀測驗\;全域搜尋,局部搜尋}
\textbf{全域搜尋的題目}\;需要看完整個文章
\begin{itemize}
\item What is indicated [mentioned/stated/suggested] in the form?
\item What can be inferred from the questionnaire?
\item What is NOT mentioned in the advertisement?
\item According to the article, which of the following is true?
\end{itemize}
\end{frame}

\begin{frame}
\frametitle{局部搜尋}
\begin{itemize}
\item Who is Mary Smith? 職業名稱旁多半會有名字
\item What time does the store open on Tuesday? 搜尋有時間的部分
\item Where is the head office located? 搜尋有地點的部分
\item How much does a one-year subscription cost? 搜尋有寫價格的部分
\item Why did Leslie Adams contact Mark Kurzweiler? 注意人名, 且有聯絡原因的部分
\end{itemize}
\end{frame}


\begin{frame}
\frametitle{多篇文章}
多半在單篇就可以找到答案\\
代表性問題
\begin{itemize}
\item 問目的, 大多在標題或開頭幾行\\
What is the purpose of this notice?
\item 題目出現NOT, 多用消去法, 可以稍後答題 \\
What is NOT mentioned in the advertisement?
\item 問同義詞, 必須從文章前後文去判斷\\
In paragraph 1, line 1, the word "affluent" is closest in meaning to?
\item 問具體內容\\
Who is Ralph Jacobsen?

\item 問文章類型, 正解通常是說明書等手冊\\
Where would these directions most likely be found?
\end{itemize}
\end{frame}
%------------
%-- Finish --
%------------
\end{CJK*}
}
\end{document}
%__ Finish __
%____________

%========= Document =========
%============================